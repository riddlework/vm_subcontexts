
\documentclass{article}

\usepackage{amsmath}
\usepackage{amssymb}
\usepackage{parskip}
\usepackage{geometry}
    \geometry{letterpaper, margin=1in}
\usepackage[shortlabels]{enumitem}

\renewcommand{\epsilon}{\varepsilon}
\renewcommand{\phi}{\varphi}
\newcommand{\N}{\mathbb{N}}
\newcommand{\Z}{\mathbb{Z}}
\newcommand{\Q}{\mathbb{Q}}
\newcommand{\R}{\mathbb{R}}
\newcommand{\st}{\;\ifnum\currentgrouptype=16 \middle\fi|\;}


\title{Design Document: Matchmaker Subcontext for Virtual Memory}
\author{Maria Fay Garcia}
\date{}
\begin{document}

\maketitle


\section{Overview}
In implementing virtual memory subcontexts, we have thus far implemented a
client-side library and a server-side library. Processes wishing to snapshot
their memory pick up the server-side library, map their memory into an image
file, and die. Processes wishing to map a specific subcontext into their memory
pick up the client-side library and select an image file to map into their
memory. Client processes can then call functions that reside within the memory
of the server process.

\section{Motivation and Problem Statement}
Clients want to map image files of server processes' memory in order to call
functions that reside within the server processes' virtual memory. To ensure
that this entry point is well-defined and safe, we will create a "Matchmaker"
subcontext which we will map into every client processes' memory before they
can map any server image files.

\section{Aims}
The Matchmaker subcontext will serve two primary purposes:
\begin{enumerate}[(1)]
    \item Act as an observer and/or mediator in the "handshake" between the client
    process and the memory of the server process.
    \item Provide a well-defined entry point into the server processes' memory when
    the client process wishes to make function calls.
\end{enumerate}

\section{Client-Server Interaction}
After initialization (i.e., mapping the Matchmaker) a client process will
request that the Matchmaker map image files into its memory. The Matchmaker, in
turn, will perform the necessary checks (if the image file exists, if the
client has permission to map the image file, etc.) before mapping the image
file into the client processes' memory.

After mapping a given server image process into the memory of a client process,
the Matchmaker will provide an interface through which the client process can
make function calls into the memory of the server process. The Matchmaker will
maintain a list of the server processes' function pointers and will provide
this list (or perhaps an abstraction of this list?) to the client process upon
mapping. When the client process requests to call a server process' function,
it again checks permissions and facilitates or rejects the call. In this way,
the Matchmaker enforces protections, both over the mapping of new server image
files and the calling of functions within already-mapped server image files.

\newpage
\section{A Tentative API}
\begin{enumerate}[(1)]
    \item \verb|init()|: Maps the Matchmaker into the client process' memory.
    \item \verb|request_map(image_filename)|: Asks the Matchmaker to map a server image of
        the given image file name into its memory.
    \item \verb|request_call(server, func_ptr)|: Asks the Matchmaker to call the function with
        the given pointer in the server process' memory.
    \item \verb|finalize()|: Asks the Matchmaker to unmap all mapped server images from the
        client process' memory. The Matchmaker does this and then unmaps itself. This function
        should be called upon finalization of the client program.
\end{enumerate}


\section{Requirements}
\begin{enumerate}[(1)]
    \item If a process picks up the client-side library, it should connect to the
    Matchmaker automatically. In other words, Once a client picks up the
    client-side library, it should not be allowed to map any image files into
    its memory before it has mapped the Matchmaker.
    \item A client process should not be able to make raw function calls directly
    into the memory of a server process. Function calls should either enter at a
    well-defined point through the Matchmaker or be mediate by the Matchmaker in
    such a way that they are definitely safe to complete.
\end{enumerate}

\section{Security and Isolation}
The existence of the Matchmaker will ensure that client processes are not
granted god-like power over the memory of server processes which they map.
Interaction with server processes happens via the Matchmaker, which will have
unrestricted permissions and will be able to modify the permissions of the
client process as it sees fit.

Ideally, a function call from a client process into a server process will cause
a segmenation fault, which we will then allow the Matchmaker to handle by
registering the SEGV signal handler. The Matchmaker will check the permissions
of the client and turn on/off permissions as necessary to allow the client to
make the desired function call or to reject the client's access to that part of
the server's memory.

\section{Concurrenchy and Synchronization}
Multiple clients should be able to map the same server image at the same time.
Clients should never modify the memory belonging to a server image. To
keep track of which clients are mapping which images, the client-side library
should maintain a global list of mapped subcontexts.

\section{Questions about Failure Handling}
\begin{enumerate}[(1)]
    \item What if an image file is missing, or the mapping of an image file
        overlaps with the memory of the client process? For now, we bail out,
        but we should consider taking a different action in the future.
    \item Should we provide cleanup routines for zombie clients?
\end{enumerate}

\newpage
\section{Questions about Constraints and Assumptions}
What limitations, if any, should we enforce on:
\begin{enumerate}[(1)]
    \item The number of subcontexts a single client is allowed to map
    \item The maximum size of an image file that a server is allowed to create
    \item The number of function pointers a server process is allowed to pass to the Matchmaker
\end{enumerate}

\section{Future Work}
\begin{enumerate}[(1)]
    \item Enforce permissions restrictions by using mprotect and registering the SEGV
    signal handler.
    \item Transition from a shared library to a static library by building an ELF
    executable by hand.
\end{enumerate}
\end{document}

